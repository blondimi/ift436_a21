\documentclass{article}

%% Packages
\usepackage[utf8]{inputenc}
\usepackage[french]{babel}
\usepackage{amsmath,amssymb}
\usepackage{enumerate}
\usepackage{multicol}
\usepackage[noline, onelanguage, french, noend]{algorithm2e}

%% Macros
\newcommand{\N}{\mathbb{N}}
\renewcommand{\O}{\mathcal{O}}

%% Titre
\title{IFT436: devoir 1}
\author{Foo McBar}
\date{21 septembre 2021}

\begin{document}
\maketitle

%% Contenu
\section*{Question 1}

\begin{enumerate}[(a)]

\item Voici mon algorithme...

    \begin{algorithm}[H]
      \DontPrintSemicolon

      \While{vrai}{
        \If{$x < y$}{
          $x \leftarrow x + 1$\;
          $y \leftarrow 2x$\;
        }
        \ElseIf{$x > y$}{
          $x \leftarrow x - 1$\;
        }
        \Else{
          $x \leftarrow 0$\;
        }
      }

      $t \leftarrow [\,]$\;

      \For{$i \in [1..n]$}{
        $t[i] \leftarrow n - i$\;
      }

      \textbf{ajouter} $42$ à $t$\;
    \end{algorithm}

\item ...

\item ...

\item ...

\item ...

\item $(m \cdot n^m \cdot 2^n)! - 9000$

\end{enumerate}

\section*{Question 2}

\begin{enumerate}[(a)]

\item Expression mathématique centrée: \[ [(0, 5), (1, 12), (8, 11),
  (3, 8), (12, 14), (13, 15), (18, 20), (1, 16)] \]

\item Un tableau:

  \begin{center}
    \begin{tabular}{lcr}
      gauche & centré & droite \\

      foo & bar & baz
    \end{tabular}
  \end{center}

\item ...

\item $i_1, i_2, \ldots, i_n$

\end{enumerate}

\section*{Question 3}

\begin{enumerate}[(a)]

\item

  \begin{multicols}{2}
    \begin{enumerate}[(i)]
    \item 42
    \item 43
    \item 44
    \item 45
    \item 46
    \item 47        
    \end{enumerate}
  \end{multicols} 

\item $n^3$, $3^n$, $n \log_2 n$, $n^2$

\item Posons $f(n) = 2(n - 4)(n + 2)(n - 3) + 8 \log(n)$. Nous avons:
  \begin{align*}
    f(n)
    &= 2(n - 4)(n + 2)(n - 3) + 8 \log(n) \\
    &\leq ... && \text{(pour tout $n \geq 5$)}  \\
    &\leq ... && \text{(pour tout $n \geq 2$)}  \\
    &= c \cdot n^3 && \text{(par magie)}
  \end{align*}
  Nous concluons donc que $f \in \O(n^3)$.

\item 

  \[ \O(1) \subset \O(\sqrt{n}) \subset \O(n) \subset \O(42^n) \]

\item Remarquons que $n^2 \geq 0$ pour tout $n \in \N$...

\end{enumerate}

\end{document}
